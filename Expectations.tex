\documentclass[11pt]{article}
\usepackage[margin=.9in,headsep=10pt]{geometry}
\usepackage{amsmath}
\usepackage{lastpage}
\usepackage{amssymb}
\usepackage{fancyhdr}
\usepackage[authoryear,round]{natbib}
\bibliographystyle{plainnat}
\usepackage{bold-extra}

\newcommand{\p}[1]{\vspace{1em} \noindent{\textbf{\textsc{#1~~}}}}

\pagestyle{fancy}
\fancyhead[LO,LE]{\textsc{Dyer research group}}
\fancyhead[RO,RE]{\textsc{Guidelines and expectations, \thepage~of \pageref{LastPage}}}
%\fancyhead[CO,CE]{\thepage}


\begin{document}
\section*{} %Summary
\vspace{-2em}
\p{Summary} This document is intended to provide an explicit framework for our professional interaction by making clear what my expectations are of you and facilitating a discussion of what your expectations are of me. This \emph{living document} is not set in stone, so please read through it and discuss with me and/or other group members any questions, issues, or suggested changes. As equal team members, we are all equally responsible for keeping the lines of communication open.

\section*{} %Goals
\vspace{-4em}
\p{Goals} You may have many reasons for entering our graduate program. Regardless of your exact reasons, your major goal should be learning how to be a professional scientist, and my major role as a mentor is to help you learn how to develop into a professional scientist. As a mentor, my specific goals are to:
\begin{itemize}
	\itemsep0em
	\item guide and support you on project of mutual interest
	\item help you bring your project to a conclusion through publication
	\item help you through the ups and downs of scientific inquiry
	\item provide you with experience and expertise necessary to pursue your career, whatever path you choose
	\item provide opportunities to develop technical skills, communication skills, professionalism, laboratory skills and the ability to problem solve
\end{itemize}
I will provide advice and direction on your research project, including direction in choosing and designing a thesis topic, researching the background information, planning and conducting field and/or laboratory analyses, writing and revising proposals, abstracts, and publishable manuscripts, and giving professional presentations.
I guide you in selecting/designing research projects with the intent that the results will be sufficiently new and important to merit publication, and I have selected you as a student because I am genuinely excited about your talents and potential.
Seeing a project through to publication requires \emph{enormous} commitment and self discipline.
I will freely provide help, advice and suggestions, but you have the ultimate responsibility for completing your thesis satisfactorily.

Please remember that as a student (and an academic in general) you are being evaluated during each of your interactions with faculty, writing projects, and research presentations.
I urge you to put your ``best foot forward" because you represent the University of Victoria, the School of Earth and Ocean Sciences, our research group, and me.

\section*{} %Letters of recommendation
\vspace{-4em}
\p{Letters of recommendation} I expect to write letters of recommendation for you, upon your request.
I will want to write as positive and honest a letter as possible, so keep me aware of your successes, and help me to find good things to say about you.
Let me help you fix areas in which you are not successful, and develop a professional attitude that keeps any insecurities in their proper place.
You can trust me to write a letter of recommendation for you that describes your skills and ability as positively and honestly as possible.
If you want me to write that you consistently do more than I expect, then make that effort; i.e., impress me!

\section*{} %Time
\vspace{-4em}
\p{Time}

\section*{} %Writing, Intellectual Propertty, and Authorship
\vspace{-4em}
\p{Writing, Intellectual Property, and Authorship}

\section*{} %Professionalism
\vspace{-4em}
\p{Professionalism}

\section*{} %Professional Development
\vspace{-4em}
\p{Professional Development}

\section*{} %Group Meetings
\vspace{-4em}
\p{Group Meetings}

\section*{} %Research Notebook
\vspace{-4em}
\p{Research Notebook}

\section*{} %Safety
\vspace{-4em}
\p{Safety}

\section*{} %Resources
\vspace{-4em}
\p{Resources}

\section*{} %Funding
\vspace{-4em}
\p{Funding} Part of my job is ensuring that students in good standing do not need to worry about funding.
A student in good standing is expected to:
\begin{itemize}
	\itemsep0em
	\item maintain a minimum GPA of 7.0 (A--)
	\item maintain progress on your research
	\item do an agreed amount of TA work
	\item make contributions to teaching when requested by me
	\item make contributions to proposals when requested by me
	\item attend and contribute to group meetings
	\item behave in a professional and collegial manner
	\item adhere to the guidelines in this document
\end{itemize}
Graduate funding in SEOS at UVic is complicated.
There is a significant amount of variation in stipend provided to students within SEOS, with a minimum stipend of \$21,000.
I pay students in excess of the minimum and attempt to do so in as fair and rational way as possible.
This document lays out some of the funding complexity and my reasons for doing things a particular way, but the \textbf{bottom line is that you will receive a minimum of \$20,064, plus the cost of tuition and fees, per year} (e.g. minimum of \$27,510 for a domestic student during the 2021-2022 academic year).
Where does the number for the minimum of \$20,064 come from? I use the StudentAid BC  \emph{moderate standard of living allowance} for a \emph{single student living away from home.}


In general, funding provided as a fellowship is not subject to tax, whereas money provided as TA salaries is subject to tax.
Typically 80\% or more of your stipend will be in the form of fellowships, and the marginal tax rate on low taxable incomes is small, so your overall tax rate is very low.

If you are successful in winning a large external fellowship (e.g. NSERC PGS, CGS, Vanier), then I believe you should be able to celebrate your success and feel rewarded.
Rather than offset the amount of money that I pay from my grants to support you, these external funds will likely mean that you will be paid excess of my standard amount for the duration of those awards (well done – and remember to treat your fellow students when you go out).

The graduate tuition structure is complicated, and the following is my understanding of the regulations:
\begin{itemize}
	\itemsep0em
	\item One tuition or reregistration fee installment is paid per term. For graduate students, there is no relation to the number classes taken.
	\item Masters students pay in order: 5 tuition installments as a combination of full or half installments, 1 full installment, followed by reregistration fees up to the time limit, then program extension fees. Minimum 5 installments.
	\item Doctorate students pay in order: 7 tuition installments, 1 half installment, 1 full installment, 1 half installment, followed by reregistration fees up to the time limit, then program extension fees. Minimum 7.5 installments.
	\item Students who transfers from Masters to Doctorate may transfer up to 5 tuition installments.
	\item Additionally, student (ancillary) fees are charged for Extended Health, Dental, Athletics, Bus Pass, and Grad Students Society membership. Basic healthcare is provided free as a right to residents of British Columbia.
\end{itemize}

The following table summarizes the calculations behind your annual stipend.
Note that if the numbers in this document do not match the public numbers at any point, your stipend and this document will be updated to reflect the new numbers as soon as possible.
For example, your base moderate standard of living allowance will reflect new StudentAid BC number when they are published.
\begin{center}
	\begin{tabular}{lrrr}
		\textbf{University Expenses}                                    & \textbf{Domestic} & \textbf{International}   \\
		\underline{per term tuition or registration expenses}           &                   &                          \\
		~~Tuition Installment                                           & \$2,052           & \$2,588                  \\
		~~Reregistration                                                & \$815             & \$1,010                  \\
		\underline{per year student fees}                               &                   &                          \\
		~~Graduate Student Society Fees                                 & \$246             & \$246                    \\
		~~Athletics                                                     & \$276             & \$276                    \\
		~~Bus                                                           & \$243             & \$243                    \\
		~~Dental                                                        & \$285             & \$285                    \\
		~~Health                                                        & \$409             & \$409                    \\
		~~BC MSP Health Fee                                             & \$0               & \$900                    \\
		~~~~{subtotal}                                                  & \$1,459           & \$2,359                  \\
		                                                                &                   &                        & \\
		\textbf{Sum of tuition and/or reregistration + fees per year}   &                   &                          \\
		\underline{Masters}                                             &                   &                          \\
		~~Year 1                                                        & \$7,615           & \$10,123                 \\
		~~Year 2                                                        & \$7,615           & \$10,123                 \\
		~~Year 3                                                        & \$3,905           & \$5,390                  \\
		\underline{Doctorate}                                           &                   &                          \\
		~~Year 1                                                        & \$7,615           & \$10,123                 \\
		~~Year 2                                                        & \$7,615           & \$10,123                 \\
		~~Year 3                                                        & \$6,589           & \$8,829                  \\
		~~Year 4                                                        & \$4,115           & \$5,674                  \\
		~~Year 5                                                        & \$3,905           & \$5,390                  \\
		                                                                &                   &                        & \\
		\textbf{Moderate standard of living allowance}                  &                   &                          \\
		~~Annual derived from StudentAid BC guidelines                  & \$20,328          & \$20,328                 \\
		                                                                &                   &                        & \\
		\textbf{Total stipend (university expenses + living allowance)} &                   &                          \\
		\underline{Masters}                                             &                   &                          \\
		~~Year 1                                                        & \$27,943          & \$30,451                 \\
		~~Year 2                                                        & \$27,943          & \$30,451                 \\
		~~Year 3                                                        & \$24,233          & \$25,718                 \\
		\underline{Doctorate}                                           &                   &                          \\
		~~Year 1                                                        & \$27,510          & \$30,451                 \\
		~~Year 2                                                        & \$27,510          & \$30,451                 \\
		~~Year 3                                                        & \$26,917          & \$29,157                 \\
		~~Year 4                                                        & \$24,443          & \$26,002                 \\
		~~Year 5                                                        & \$24,233          & \$25,718                 \\
		                                                                &                   &                        & \\
	\end{tabular}
\end{center}




\newpage
\section*{} %Acknowledgement
\vspace{-4em}
\p{Acknowledgement} This document was inspired by similar documents developed by Kate Huntington (University of Washington) and Colin Goldblatt (University of Victoria). Also, I would like to extend a special thank you to Christine Chen (postdoc, Caltech) for making me aware of the value and importance of this document.
\section*{} %Version
\vspace{-4em}
\p{Version} This document was last updated on \today{}.





\end{document}
